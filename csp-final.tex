%!TEX program = xelatex
% 完整编译: xelatex -> bibtex -> xelatex -> xelatex
\documentclass[lang=cn,11pt,a4paper,cite=authoryear]{elegantpaper}

% 本文档命令
\usepackage{array}
\usepackage{authblk}
\newcommand{\ccr}[1]{\makecell{{\color{#1}\rule{1cm}{1cm}}}}

\title{大数据场景下语言虚拟机的性能分析与优化调研}

\author{张正君, 余博识, 贾兴国 \\ Shanghai Jiao Tong University \\ \{zhangzhengjunsjtu, 201608ybs, jiaxg1998\}@sjtu.edu.cn}

\institute{\href{http://www.se.sjtu.edu.cn/}{School of Software Engineering}}

\version{1.0}
\date{\zhtoday}



\begin{document}

\maketitle

\begin{abstract}
随着大数据处理系统的快速发展,托管型语言(managed languages)由于其快速的开发周期以及丰富的社区资源得到了广泛的应用。例如,分布式数据处理框架Spark、Hadoop、DryadLINQ、分布式文件系统HDFS、分布式键值存储系统HBase、Cassandra等均使用开发效率较高的Java语言进行开发。托管型语言通过使用语言虚拟机实现对内存的管理,如使用JVM(Java Virtual Machine)进行垃圾回收(garbage collection,GC),免去手动编写代码进行内存释放的需要,也降低了内存管理出现错误的可能。然而,开发的便利性也带来了一定的性能损耗。经过调研,研究者们从不同的角度发现了托管型语言在分布式大数据处理场景下出现的性能问题,如GC时间过长、节点间数据传输太慢、语言虚拟机启动耗时过长等,并且提出了相应的解决方案。本文总结了研究者们发现的问题及其解决方案,并且针对解决方案的局限性与适用性进行了分析,提出了相应的优化方案。
\keywords{语言虚拟机,分布式系统,大数据,垃圾回收}
\end{abstract}

\cleardoublepage
\tableofcontents
\cleardoublepage

\section{研究背景}
随着摩尔定律的终结,单个机器所提供的计算资源已经无法满足海量数据的巨大算力需求。学术界和工业界将关注点从纵向扩展转向了横向扩展,使用多台配置较低的廉价机器代替了配置较高的单台机器,使用分布式系统向数据处理、机器学习、云计算等应用提供可横向扩展的计算资源。运行在分布式系统上的数据处理框架如Spark\cite{spark}、Hadoop\cite{hadoop}、DryadLINQ\cite{DBLP:conf/osdi/YuIFBEGC08}、Naiad\cite{DBLP:conf/sosp/MurrayMIIBA13}等采用了分布式并行的数据处理模式,充分利用分布式集群提供的计算资源。也有分布式文件系统HDFS\cite{hdfs}、分布式键值存储系统HBase\cite{hbase}、Cassandra\cite{cassandra}等利用分布式系统较强的容错性、可扩展性,为数据存储提供了更优的解决方案。

运行在云平台上的分布式软件的快速发展使得托管型编程语言成为了开发人员和研究人员的关注重点。广泛使用的分布式处理框架Hadoop、Spark、Cassandra均使用Java进行开发,DryadLINQ、Naiad等数据处理框架使用由微软开发的C\#语言进行编写。还有更多的托管型语言被广泛使用,如Go、Javascript、Python等。托管型语言如Java、C\#使用语言虚拟机进行自动内存回收,免去了手动编写代码释放内存的需要,还有面向对象的编程风格、丰富的社区资源,使得Java等语言有很高的开发效率。

语言虚拟机带来的开发便捷性建立在一定的性能损耗之上。例如为了运行Java程序,首先对Java源代码进行编译,生成Bytecode(字节码),再从classpath中加载、解析、验证字节码文件,以及运行该字节码所需的其他字节码文件,后在JVM中运行字节码。Java语言运行环境(runtime)的加载以及字节码的解释执行相比于Native的编程语言(如C、C++)产生了巨大的开销。同时,Java程序在堆(heap)内存中为对象申请内存,无需手动编写代码释放堆内存,而是使用JVM自带的GC(垃圾回收)功能进行对内存的自动释放。然而,进行GC的过程中,语言虚拟机需要占用应用的CPU资源,或需要应用暂停执行(Stop-the-world),这在延迟敏感的情境下对应用的性能表现有巨大的影响。在分布式数据处理的场景中,由于Java等管理型语言使用在堆上保存的对象(Object)来存储、处理数据,GC需要遍历海量的对象,其开销相对于单机、数据量较小的情景也被明显加大。分布式环境下请求的处理依赖于多个节点上的语言运行环境,语言虚拟机的性能优化需要依靠多个语言虚拟机之间的协调,这和单机情境下的语言虚拟机性能优化又大有不同。

本文针对大数据处理场景下的语言虚拟机性能优化进行调研,总结出了分布式大数据处理对语言虚拟机提出的新的挑战,以及研究者所提出的解决方案,并分析了特定解决方案的适用性与优化方案。本文第二章总结了语言虚拟机在大数据情境下表现出的性能问题,以及研究者对相关问题所提出的解决方案。第三章就JVM启动耗时过长问题的研究思路进行了分析,并总结了其局限性与适用范围。第四章提出了针对第三章中适用性问题的解决方案,并搭建原型、验证和测试了解决方案的有效性。在第五章中我们对本文的调研工作进行总结。

\section{大数据场景下语言虚拟机的性能问题与优化}
大数据处理为托管型语言带来了全新的使用场景,同时托管型语言为大数据处理框架带来了较高的开发效率。新的使用情景为托管型语言的语言虚拟机带来了新的挑战。本章对大数据环境下语言虚拟机表现出的性能问题做了总结分析,并对研究者的相应解决方案进行了总结。

\subsection{分布式环境下GC对性能的影响}
\subsubsection{问题概述}
在大数据场景下,应用需要良好的性能与扩展性,然而托管型语言的运行时环境存在臃肿、可扩展性差的问题。语言虚拟机的GC使得应用的主线程必须停止,需要遍历堆中保存的相互引用的复杂的对象,占用大量计算资源,无法进行有用的工作;内存管理模块也有巨大的额外内存开销,虽然所处理的数据量大小比堆内存的容量更小,也会产生OOM(Out Of Memory,内存不足)的报错。但使用非管理型语言会增加应用编写者的负担,更容易出现错误,调试内存管理错误也是一件非常困难的事。当前大量的大数据处理框架已经使用管理型语言进行编写,向非管理型语言的迁移会带来巨大的工作量。故需要有一种方法优化语言虚拟机的GC问题与内存空间占用问题。

\subsubsection{解决方案}
Nguyen等人提出了Facade\cite{DBLP:conf/asplos/NguyenWBFHX15}编译框架。他们认为将大量数据保存在堆内存上,让JVM进行管理是不明智的。通过限制堆内存上保存的对象数量,将数据面与控制面进行分离,仅在堆上保存控制面对象,而数据面中的数据保存在本地内存(native memory)中进行手动管理。这打破了面向对象编程语言的一条规则:对象应当包含数据以及操作数据的接口。对于每一个Java语言中的堆上保存的对象,Facade编译系统生成一个相应的facade,提供对象的操作接口而不保存对象中包含的数据,从而减轻垃圾收集器的负担,而程序员仅需标注出数据类(data class)。对于facade,可以通过对象复用限制堆中对象的数量;而native memory中的数据在每个iteration结束后统一进行释放,无需访问数据的每个字段,相比于垃圾收集器有更优的性能。经测试,Facade使得应用运行时间缩短了48\%,降低了50\%的内存占用。

Gog等人提出了Broom\cite{DBLP:conf/hotos/GogGSVVRCMHI15}内存管理系统,取代了CLR的GC系统。他们认为,大数据系统中存在多个Actor,这些Actor独立地运行,它们之间传递消息,形成了高度结构化的数据流,例如MapReduce\cite{DBLP:journals/cacm/DeanG08}中的map工作线程与reduce工作线程。每个Actor内的对象在Actor完成工作后不再会用到,仅有Actor之间传递的数据需要保存,无需GC扫描。为了针对这类高度结构化的应用对垃圾收集器进行优化,Broom内存管理器使用了基于region(分区)的内存管理策略,对象在各个region中保存而不在堆中由GC管理,由程序员手动释放region中的对象。Broom内存管理器减轻了GC的负担,减少了59\%的运行时间,但使得程序员的负担增加。

\subsection{JVM启动耗时对延迟敏感型应用的影响}
\subsubsection{问题概述}
社区中长期存在对Java性能的激烈讨论,尤其是Java是否适合编写延迟敏感型应用的问题。一些程序员坚持使用C++等非管理型语言编写键值存储系统,因为他们认为Java天生就是较慢的。Java之所以适用于编写Hadoop框架是因为Hadoop大部分时间在进行I/O操作。在学术界,对于大数据框架的性能优化主要集中在对GC性能的优化、调度策略的优化、数据交换开销的优化等,故我们缺少对Java各个方面性能的整体了解。经过实验测得,在I/O密集型任务中,JVM的预热开销也是较大的,如在从HDFS中读取1G的文件的过程中,JVM预热使用的时间占总运行时间的33\%,预热时间不随任务数据量的变化而变化。在大数据框架中,复杂的软件栈使得Java运行环境的加载更加缓慢,因为需要加载更多的class,如Spark为了完成一次请求需要加载19066个class。

\subsubsection{解决方案}
Lion等人经过实验测得,JVM的性能瓶颈在于JVM的启动预热(warm-up),并编写了HotTub代替了HotSpot\cite{DBLP:conf/osdi/LionCSZGY16}解决了Java虚拟机启动时间过长的问题。他们首先通过修改Java程序的调用过程测量出class加载与字节码解释执行的开销,观察出JVM预热对性能的巨大影响。同时,他们在HotSpot的基础上修改实现了HotTub,在多次请求中复用同一个JVM实例,从而免去JVM的预热开销。由于大数据应用的相似性,JVM有较大的复用可能,同时频繁使用的字节码在多次复用后被JIT编译成机器码而无需解释执行。HotTub可以让Java程序无修改地运行,免去了JVM的预热开销,使得Spark请求的性能提升到1.8倍。本文将在第三章对Lion等人的HotTub研究进行详细的分析,总结其使用范围与局限性,并在第四章中提出并测试相应的优化方案。

\subsection{分布式环境下多个JVM的协调问题}
\subsubsection{问题概述}
在分布式大数据处理框架中,工作负载运行在多个节点上,即运行在多个相互独立的运行时系统(语言虚拟机)上。这会成为性能降低的原因之一,因为每个运行时系统仅拥有当前运行节点的运行状况,而非全局的分布式应用的运行状况。这是之前的工作无法解决的问题,之前的工作仅针对单个节点上的运行时系统进行性能优化。例如,某个节点在不合适的时间进行GC,导致该节点成为整个请求的拖后腿者,虽然其余节点在较短时间内完成了自己的工作,但仍需等待最慢的节点完成GC,才能将请求结果返回给用户。这将会对延时敏感型与面向吞吐量的应用都造成性能影响。
\subsubsection{解决方案}
Maas等人设计的Taurus\cite{DBLP:conf/asplos/MaasA0K16}系统通过在分布式处理系统中加入一个分布式决策机制,将每个节点上独立的运行时系统组织为一个Holistic Runtime System(全局运行时系统),从而做出在分布式系统全局下正确的GC、JIT决策。当每个Holistic Runtime System中的runtime实例启动时,

\subsection{分布式环境下节点数据交换的性能问题}
\subsubsection{问题概述}
在大数据系统中,节点间数据的传递是频繁发生的。和之前GC性能问题的根源相同,数据传递也因为需要处理海量的包含着数据的对象(Object)而受到了性能影响。数据发送方需要对于Java中的对象(Object)进行\textit{serialize}(序列化)才可以在网络中传递,接收方需要将接收到的字节流进行\textit{deserialize}(反序列化)才能生成发送方想要发送的对象。序列化和反序列化严重依赖于Java中的反射机制,这是一个开销较大运行时操作,会严重影响应用的性能;同时程序员需要手写序列化和反序列化函数,较容易出现错误。这要求我们进一步考虑Java等面向对象语言中Object与数据处理的关系。
// TODO 完善
\subsubsection{解决方案}
Nguyen等人提出的Skyway\cite{DBLP:conf/asplos/NguyenFNXDL18}将本地与远程的JVM进程中的堆进行连接,使得源节点的堆中的对象无需序列化即可传输到目的节点 // TODO 完善

Navasca等人认为,数据分析的任务经常使用的数据类型是不可变且被限制的,他们开发的Gerenuk\cite{DBLP:conf/sosp/NavascaCNDLKX19}编译器 // TODO 完善

\subsection{资源分散架构下的GC算法优化}
\subsubsection{问题概述}
在大数据处理集群中,资源碎片化问题影响着集群资源使用率,如内存资源,单个节点不一定在全部的时间段都对内存有较高的利用率,当内存利用率较低时,// TODO 待完善
\subsubsection{解决方案}
Chenxi Wang等人开发的Semeru\cite{semeru}是一个分布式的JVM // TODO

\section{HotTub研究思路分析}
\subsection{JVM预热开销测试与分析}
// TODO
\subsection{HotTub设计与实现}
// TODO
\subsection{HotTub性能测试}
// TODO
\subsection{HotTub适用性与局限性分析}
// TODO

\section{HotTub局限性优化}
// TODO

\section{总结}
// TODO

\nocite{*}
\cleardoublepage
\bibliography{wpref}

\appendix
%\appendixpage
\addappheadtotoc
\section{附录}

\subsection{//}




\end{document}
